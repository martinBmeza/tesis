\section[Líneas futuras de investigación]{CAPÍTULO 7:$\ \ \ \ $LÍNEAS FUTURAS DE INVESTIGACIÓN} 

Si bien los resultados obtenidos en este trabajo son satisfactorios y demuestran el potencial del uso de redes neuronales profundas y técnicas de aumentación de datos, algunas limitaciones fueron notadas. Por un lado, la información de fase fue omitida, y trabajos recientes sugieren que pueden obtenerse mejoras expandiendo el procesamiento para que se consideren también las componentes complejas de la STFT, es decir, realizar la dereverberación en amplitud y fase. Esto se podría conseguir estimando máscaras complejas o bien estimando máscaras independientes para magnitud y fase. Por otro lado, en este enfoque se procesa de la misma manera la totalidad del espectro. Sabiendo que la reverberación aporta más energía en bajas frecuencias, podría segmentarse el espectro en bandas y procesar cada banda con configuraciones diferentes, o utilizar otras funciones de costo que compensen la energía en alta frecuencia. De manera más general, también podrían modelarse de forma eficiente dependencias temporales presentes en la reverberación introduciendo capas recurrentes.

Otro aspecto a considerar es el referido a las métricas utilizadas tanto para el entrenamiento como para la evaluación de los modelos. Se deben analizar y proponer nuevas métricas que se correlacionen de manera directa con la percepción auditiva de la tarea de dereverberación \cite{CDPAM}. De esta forma el entrenamiento se realizará en pos de una mejora en la percepción auditiva de los resultados, y los resultados de las evaluaciones podrán ser utilizados para tomar decisiones que mejoren la calidad del proceso de dereverberación de manera consistente y objetiva.

Respecto al proceso de aumentación, es de interés tener un cierto control sobre el perfil espectral del tiempo de reverberación con el que se generan las respuestas al impulso aumentadas. En este trabajo, las respuestas al impulso aumentadas poseen el mismo perfil espectral de TR que las originales. Una mayor diversidad de respuestas podría conseguirse si se logra controlar también la forma general del perfil de tiempo de reverberación. 

Perfeccionar las técnicas de aumentación de respuestas al impulso y recolectar nuevas podría llevar en un futuro a la creación de un corpus de datos de dominio libre destinado específicamente al desarrollo de sistemas de dereverberación del habla, estandarizando los métodos de evaluación de los estudios que se realicen en el área. Generar esta base de datos de acceso libre puede asegurar que exista una correcta comparación entre las soluciones propuestas por diferentes trabajos, de forma tal que cada resultado logre contribuir de manera objetiva a la mejora general de las técnicas de dereverberación. 