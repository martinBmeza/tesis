\section[Conclusiones]{CAPÍTULO 6:$\ \ \ \ $CONCLUSIONES} 

En este trabajo se implementó un algoritmo de aprendizaje profundo para la tarea de dereverberación de señales de habla. Se utilizó una estructura de tipo autoencoder con conexiones de salto, la cual se entrenó para estimar máscaras de amplitud que al aplicarse sobre un espectro de magnitud reverberado generen un espectro de magnitud dereverberado. Luego, para obtener la señal temporal a partir de la magnitud del espectrograma dereverberado, se utilizó la fase del espectrograma reverberado y se invirtió mediante la transformada inversa de Fourier y la técnica de suma y solapamiento.
 
Principalmente, se puso foco en analizar la influencia de ampliar el conjunto de datos de entrenamiento con técnicas de aumentación y síntesis de respuestas al impulso. También se analizó el efecto del ordenamiento de los datos de entrenamiento, y de la técnica utilizada para la inversión del espectrograma dereverberado.

Se pudo comprobar que una mayor diversidad de respuestas al impulso para la generación de los datos de entrenamiento genera mejores resultados, y que tanto las respuestas al impulso generadas como las aumentadas son útiles como método de aumentación de datos de entrenamiento. Por otro lado, se pudo comprobar que aplicar aprendizaje por currículum, ordenando los datos con un TR creciente, permite obtener mejores resultados de dereverberación.
 
Estos resultados son un aporte importante para el estudio de las tareas de dereverberación de audio, para las cuales las bases de datos de respuestas al impulso existentes son escasas, poco diversas y en ocasiones difíciles de conseguir o muy costosas.