
\renewenvironment{abstract}
 {\par\noindent\textbf{\abstractname}\ }

\renewcommand{\abstractname}{\normalfont\fontsize{16}{18}\bfseries RESUMEN\vspace{14pt}}

\begin{abstract}

En este trabajo se estudia la dereverberación de señales de habla a partir de algoritmos de aprendizaje profundo. Se implementa una de red neuronal convolucional tipo autoencoder con conexiones de salto, basada en el estado del arte actual, para estimar máscaras de amplitud que realicen la dereverberación del habla en el dominio de la transformada de tiempo corto de Fourier. Uno de los problemas de esta tarea es la escasa cantidad de datos disponibles, por lo que se analizan técnicas de generación y aumentación de datos, evaluando su impacto en el desempeño del sistema. Además, se evalúa el efecto que tiene el ordenamiento de los datos de entrenamiento y el tratamiento de la información de fase. Los resultados indican que las técnicas de generación y aumentación de datos permiten mejorar el rendimiento final del sistema. A su vez, ordenar los datos de entrenamiento con un tiempo de reverberación creciente tuvo un impacto positivo en las métricas de evaluación. Por último, se proponen mejoras al enfoque utilizado, y líneas futuras de investigación.

\vspace{14pt}

\textbf{Palabras clave}: ``Dereverberación del habla"; ``Redes Neuronales Convolucionales"; ``Respuestas al Impulso"
\end{abstract}

\newpage

\renewcommand{\abstractname}{\normalfont\fontsize{16}{18}\bfseries SUMMARY\vspace{14pt}}
\begin{abstract}

In this research, speech deverberation based on deep learning algorithms is studied. A convolutional neural network model called autocoder with skip connections is implemented, following the current state of the art. The neural network is developed to estimate amplitude masks that perform speech deverberation in the short-time Fourier transform domain. One of the issues of this particular task is the lack of a large training dataset, so generation and augmentation techniques were analyzed, evaluating the impact on model's performance. In addition, ways to handle phase information and data during training were studied. The results show that the generation and augmentation techniques allow to improve the model's performance. Moreover, sorting the training data from smallest to largest reverberation time results in better evaluation metrics. Finally, improvements for the implemented model and future lines of research were proposed.

\vspace{14pt}

\textbf{Keywords}: ``Speech dereverberation"; ``Convolutional Neural Networks"; ``Room Impulse Response"
\end{abstract}
