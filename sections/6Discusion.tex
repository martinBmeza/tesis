\section{Análisis de los resultados}

\subsection{Analisis de bases de datos de respuestas al impulso}
A lo largo de los experimentos se generó audio de habla reverberado a partir de respuestas al impulso reales, generadas y aumentadas. Del análisis de estos conjuntos utilizando como medida la relación directo reverberado  y el tiempo de reverberación medio se pueden observar algunas particularidades. Respecto a las respuestas al impulso reales, en la figura \ref{fig:cloud_reales} se pueden distinguir tres grandes grupos de puntos de acuerdo a los tres recintos de los cuales fueron obtenidas dichas respuestas. Las variaciones de ambos parámetros acusticos ocurren debido a las diferentes posiciones de microfono que han sido utilizadas en cada recinto, produciendo variaciones de DRR en un rango de $10 dB$ y de tiempo de reverberación de $2 \ s$ aproximadamente. Pese a estas variaciones, la distribución de los puntos no es uniforme ni abarcativa, sin mencionar que no exiten respuestas para tiempos de reverberación menores a $1.8 \ s$. Algo similar ocurre con las respuestas al impulso generadas que se muestran en la figura \ref{fig:cloud_generadas}. Si bien en este caso se tiene control sobre los puntos centrales de los conjuntos de puntos (se generaron para tiempos de reverberación de $0.5 \ s$, $0.75 \ s$ y $1.0 \ s$) ocurre el mismo fénomeno que con las respuestas al impulso reales, en donde se forman grupos de puntos que no se dispersan uniformemente en el plano. Esto cambia para el tercer conjunto que corresponde a las respuestas al impulso generadas a partir del proceso de aumentación. La dispersión de estas respuestas se observa en la figura \ref{fig:cloud_aumentadas}. A primera vista se observa una mayor uniformidad de los puntos en el plano, ya que no se aprecian conjuntos separados sino mas bien una aleatoriedad uniforme a lo largo del rango del rango generado. La uniformidad de la dispersion la podemos atribuir al control que se tiene sobre estos parámetros a la hora de generar las respuestas aumentadas, y la aleatoriedad entre los puntos se debe al hecho de que siempre se parte de una respuesta al impulso real diferente para realizar la aumentación, lo cual produce que el margen entre los parametros deseados y los obtenidos sea variable. Por otro lado, parece haber una menor cantidad de puntos en la esquina inferior izquierda del grafico, es decir, tiempos de reverberación bajos con relaciones directo-reverberado bajos. Esta es una limitación tanto propia del algoritmo de aumentación como tambien de la naturaleza de las respuestas al impulso reales, en donde para tiempos de reverberación bajos la energia de la parte tardia de la respuesta es de por si baja. 

\subsection{Funcionamiento general del sistema}
La figura \ref{fig:mean and std of nets} muestra el funcionamiento general del sistema. Se puede ver que el aprendizaje de la red neuronal desemboca en la generación de máscaras de amplitud que mantienen la ganancia de las componentes propias del habla anecoica y atenuan las componentes introducidas por el fenómeno de la reverberación. La atenuación de las componentes reverberantes ocurre con mayor precisión para frecuencias bajas, en donde hay mas energía. A pesar de esto, se pueden observar rasgos del espectro reverberado aun presentes en el espectro dereverberado, lo que es de esperarse debido a que el proceso únicamente esta aplicando un filtro de amplitud por sobre la magnitud del espectro reverberado.

\subsection{Determinación de fase para el espectro dereverberado}
De los resultados obtenidos sobre las distintas alternativas estudiadas para obtener el espectro de fase que complemente a la magnitud dereverberada generada se puede remarcar en primer lugar que el análisis se realiza tomando como medida las métricas objetivas utilizadas en este trabajo. Estas métricas no siempre reflejan con fidelidad lo que ocurre en la percepciíon auditiva de los resultados. Durante la realización de estas pruebas, ocurrió que el método que arrojaba mejores resultados sobre las métricas no era el que mejor se percibia auditivamente. También, ocurrio que sobre determinados ejemplos ambas alternativas generaban distorsiones subjetivamente similares pero producian resultados notoriamente distantes sobre las métricas. De igual manera, la utilización de la fase reverberada de manera directa resultó ser el método mas robusto frente a las métricas. Esto, sumado a su utilización en otros trabajos del estado del arte AGREGAR REFERENCIA hizo que esta alternativa sea la escogida a lo largo de este trabajo. 

\subsection{Funcionamiento en relacion al tipo de respuestas al impulso utilizadas}
Del primer conjunto de pruebas, lo esperado era obtener los mejores resultados para aquellos casos en los que el conjunto de entrenamiento y el conjunto de evaluación coincidian. Esto ocurrió para la evaluación con la métrica ESTOI. Para las otras métricas, el comportamiento esperado ocurrió en general para los conjuntos formados con respuestas al impulso reales y aumentadas, pero no para las generadas. Particularmente, utilizar respuestas al impulso aumentadas durante el entrenamiento produjo mejores resultados al evaluar sobre respuestas al impulso generas que usando respuestas al impulso generadas durante el entrenamiento. Esto puede deberse al hecho de que, si bien ambos conjuntos contienen tiempos de reverberación del mismo rango, las respuestas al impulso aumentadas tienen una distribución mas uniforme a lo largo de este rango.

Para el segundo conjunto de pruebas se combinaron tipos de respuestas al impulso en la conformación de los datos de entrenamiento y se volvió a evaluar en los mismos conjuntos de la primera prueba, asegurándose que el numero de instancias de entrenamiento se mantenga fijo en todas las pruebas. Cabe destacar que es de mayor importancia para éste trabajo evaluar el rendimineto al evaluar sobre respuestas al impulso reales, que es la finalidad del sistema implementado. En esta prueba, para todas las métricas los mejores resultados se obtuvieron al combinar los tres tipos de datos en la conformación del conjunto de entrenamiento. Es decir, una mayor diversidad de impulsos presentes a la hora de generar los datos de entrenamiento desemboca en una mejora en el rendimiento del sistema. Además, la combinación de respuestas al impulso reales-generadas arrojó mejores resultados que la combinacion reales-aumentadas para todas las métricas. Esto puede deberse al hecho de que las respuestas al impulso aumentadas si bien varian la pendiente de caida de la cola reverberante, mantienen el mismo perfil frecuencial que las respuestas al impulso reales de las que provienen. En este aspecto, las respuestas al impulso generadas pueden enriquecer en mayor medida la diversidad de los datos utilizados aportando nuevos perfiles de tiempo de reverberación, lo cual puede llevar a mejores resultados generales. 

\subsection{Influencia del orden de los datos durante el entrenamiento}
Se pudo corroborar el impacto del ordenamiento de los datos de entrenamiento en el rendimiento final del sistema. En este caso, la complejidad de los datos se asoció al tiempo de reverberación de cada instancia. Con esto, al entrenar con datos ordenados por tiempo de reverberación de manera creciente, es decir, iniciando con tiempos de reverberación bajos y aumentando progresivamente el tiempo de reverberación se obtuvieron mejores resultados para todas las métricas. También se comprobo que el ordenamiento inverso produce el peor resultado de los tres, y el orden aleatorio cae en un punto medio entre estos dos casos. Esto es consistente con la teoria de la técnica de aprendizaje por curriculum. En primera instancia, para tiempos de reverberación bajos, es sencillo para la red neuronal mantener el sonido directo que es predominante en la señal debido a la escasa reverberación (la máscara ideal se asemeja a una máscara unitaria). Esta información aprendida un primer momento guia en cierta medida al algoritmo ir aprendiendo de instancias progresivamente mas complejas en donde la energia proveniente de la reverberación es cada vez mayor. 
