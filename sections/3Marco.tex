\section{Marco Teórico}


\subsection{Representación temporal y frecuencial de señales}

\subsection{Respuesta al impulso y reverberación}

Si en un recinto tengo una fuente y un micrófono captando a una cierta distancia de la fuente, las ondas sonoras que emite la fuente se reflejaran en las paredes del recinto y alcanzaran el micrófono inmediatamente después que la onda sonora directa. IMAGEN. Las reflexiones continúan ocurriendo, y cada instancia de reflexión supone una disminución de la energía sonora de la onda, principalmente causada por el efecto de absorción acústica de las superficies que producen las reflexiones.En un determinado tiempo, la energía sonora decaerá en todo el recinto hasta ubicarse por debajo del ruido de fondo. A este proceso se lo denomina reverberación. Al camino mas corto entre la fuente y el punto de captura se denomina camino directo, y a la relación de nivel entre la presión sonora que genera la onda propia del camino directo y la presión que genera el efecto de reverberación se lo conoce como relación directo-reverberado. 

Si el micrófono se ubica cerca de la fuente va a captar en mayor medida la señal correspondiente al camino directo, y una pequeña porción del sonido reverberado. Es decir, una relación directo-reverberado alta. A medida que el punto de captura se aleja de la fuente va a captar una menor cantidad del sonido correspondientemente al camino directo, mientras que el campo reverberado se mantendrá aproximadamente invariante. Esto se traduce en una disminución de la relación directo-reverberado. 

De esta manera, habrá una distancia específica para la cual el nivel de presión sonora generado por la fuente sera igual al nivel de presión sonora generado por el efecto de la reverberación. Esta distancia se conoce como distancia crítica. Esta depende tanto de las condiciones del recinto como de las características del micrófono. 

La función de transferencia entre la fuente emisora y el micrófono se define como la respuesta al impulso del recinto (RIR por sus siglas en inglés) y usualmente se denota como $h(t)$. Este sera diferente para cualquier punto en el espacio dentro del recinto. Haciendo un análisis temporal de una respuesta al impulso, podemos identificar 3 partes: En primer lugar el nivel de sonido directo (producido por la onda que viaja a través de camino directo), las reflexiones tempranas (cuyo limite temporal vendrá definido por las características propias de cada recinto) y por ultimo la cola reverberante. Se puede distinguir la parte de reflexiones tempranas y la cola reverberante partiendo de la suposición de que las reflexiones tempranas ocurren en un proceso determinístico, siendo altamente sensibles a pequeños cambios en la geometría del recinto, mientras que la cola reverberante es mas bien un proceso estocástico, y al depender de un mayor número de reflexiones no varia drásticamente frente a pequeños cambios de geometría.  

Idealmente, el micrófono captura una señal que corresponde a la convolución entre la respuesta al impulso del recinto y la señal fuente.

\begin{equation}
\label{eqn:pythagorean}
	$x(t) = h(t) * s(t)$



\end{equation} 


\subsection{Inteligibilidad y parámetros de calidad de percepción}

\subsection{Redes neuronales y algoritmos de aprendizaje}


