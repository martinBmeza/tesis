 \documentclass[12pt]{article}[a4]
%\usepackage[utf8]{inputenc}
\usepackage[spanish,es-tabla]{babel}
\usepackage{amsmath,multicol}


\usepackage[backend=bibtex,sorting=none]{biblatex}
%usepackage[resetlabels,labeled]{multibib}
%\usepackage[sorting=none]{biblatex}

\addbibresource{referencias.bib}


\usepackage{fancyhdr}
\usepackage{import}
\usepackage{tikz}
\usepackage{booktabs}
\usepackage{graphicx,float,subcaption}
\newcommand{\gr}{^{\circ}}
\newcommand{\ohm}{\Omega}
\usepackage{import}
\usepackage{wrapfig}
\usepackage{url}
\usepackage{color}
\usepackage{epsfig}
\usepackage{multirow}
\usepackage{colortbl}

\usepackage{graphicx}
\graphicspath{ {imagenes/} }
\usepackage[font=small,justification=centering]{caption}
%\usepackage{subcaption}

\usepackage[left=3cm,right=2.5cm,top=2.5cm,bottom=2.5cm]{geometry}
\usepackage{tikz}
\usetikzlibrary{calc}

\usepackage{fontspec}
\usepackage{times} 
\usepackage{mathptmx}
\usepackage[font={footnotesize,it,bf}]{caption}



\usepackage{datetime}
\newdateformat{monthyeardate}{
  \monthname[\THEMONTH]  \THEYEAR}  \makeatletter
\renewcommand{\monthnamespanish}[1][\month]{%
  \@orgargctr=#1\relax
  \ifcase\@orgargctr
    \PackageError{datetime}{Invalid Month number \the\@orgargctr}{%
      Month numbers should go from 1 to 12}%
    \or Enero%
    \or Febrero%
    \or Marzo%
    \or Abril%
    \or Mayo%
    \or Junio%
    \or Julio%
    \or Agosto%
    \or Septiembre%
    \or Octubre%
    \or Noviembre%
    \or Diciembre%
    \else \PackageError{datetime}{Invalid Month number \the\@orgargctr}{%
      Month numbers should go from 1 to 12}%
  \fi}
\makeatother

\usepackage{titlesec}
\titleformat{\section}
  {\normalfont\fontsize{12}{14}\bfseries}{\thesection .}{1em}{}
\titleformat{\subsection}
  {\normalfont\fontsize{12}{14}\bfseries}{\thesubsection .}{1em}{}

\usepackage[hidelinks]{hyperref} % Para usar hipervinculos
\usepackage[]{verbatim} % Para poder comentar por bloques
\usepackage[]{lipsum} % Para rellenar con texto

\title{De-reverberación del habla a partir de algoritmos de aprendizaje profundo}

\author{Meza, Martin Bernardo}

\begin{document}
%\setlength{\parindent}{12pt}

\setmainfont{Carlito}
%\setsansfont[Scale=MatchLowercase]{Calibri}
\usetikzlibrary{calc}
\makeatletter
\begin{titlepage}
	\begin{tikzpicture}[remember picture, overlay]
  	\draw[line width = 1pt] ($(current page.north west) + (1.5cm,-2cm)$) rectangle ($(current page.south east) + (-1.5cm,2cm)$);
	\end{tikzpicture}
    \begin{center}
	   	   
	   \includegraphics[width=0.6\textwidth]{imagenes/logo.png} 
	          
        \vspace*{0.36cm}
        \Large
        \fontsize{18pt}{18pt}\selectfont\
        \textbf{INGENIERÍA DE SONIDO}
               
        \vspace{4.5cm}
        \fontsize{22pt}{22pt}\selectfont\
        \textbf{\@title \dag}
        
        \vspace{3.5cm}
        \fontsize{18pt}{18pt}\selectfont\
        \textbf{Autor: \@author }
        
        \fontsize{16pt}{16pt}\selectfont\
        \textbf{Tutores: Ing. Leonardo Pepino}
        
        \vspace{3.5cm}
        
	  \normalsize(\dag) \textbf{Tesis para optar por el titulo de ingeniero/a de Sonido}.
        
        \vspace{0.8cm}
        
        
        \large
        \monthyeardate\today
        
    \end{center}
\end{titlepage}
\makeatother
\newpage
\setcounter{page}{1}
\setcounter{section}{0}
\setmainfont{Carlito}
\begin{center}
    \textbf{Plan de Tesis}
\end{center}
\section[Introducción]{CAPÍTULO 1:$\ \ \ \ $INTRODUCCIÓN} 

\subsection[Fundamentación]{FUNDAMENTACIÓN}
Las tecnologías que explotan el procesamiento digital de señales de voz mostraron grandes avances en las ultimas décadas, llegando a ocupar roles de primera importancia en nuestro día a día. Las investigaciones realizadas en este campo fueron impulsando diversas aplicaciones basadas en el análisis de la voz humana \cite{fun1}\cite{fun2}. 
Estas tareas, en mayor o menor medida, deben lidiar con una característica intrínseca a cualquier emisión sonora dentro de un recinto: la reverberación. Las señales de voz que reciben las aplicaciones anteriormente nombradas por lo general se obtienen a partir de un transductor que no siempre se encuentra cercano a la fuente que desea registrar, provocando que la señal resultante capte la reverberación propia del entorno de origen de la señal. Esta reverberación interfiere en detrimento la señal de voz, produciendo una reducción en el rendimiento de aquellas aplicaciones que dependen de la integridad de dicha señal, como ser: 

\begin{itemize}
    \item Reconocimiento del habla \cite{reconocimiento}
    \item Verificación del hablante\footnote{Debe distinguirse entre reconocimiento del habla y verificación del hablante. Lo primero refiere a poder distinguir que palabras fueron dichas, y lo segundo refiere a identificar quien es el que esta pronunciando las palabras.} \cite{verificacion}
    
    \item Localización del hablante \cite{localizacion}
    \item Inteligibilidad de la palabra
\end{itemize}


Si bien esta problemática fue abordada desde el enfoque de diversas técnicas de procesamiento de señales, en los últimos años este campo de estudio tuvo grandes avances producto de la implementación de una tecnología emergente de amplio crecimiento en el ambiente científico: los algoritmos de aprendizaje profundo. La capacidad y robustez que esta técnica demostró a la hora de resolver problemas pertinentes al procesamiento de imágenes y detección de patrones frente a los enfoques clásicos la pusieron al frente de las herramientas utilizadas para resolver problemas de este ámbito. Sin embargo, las tareas relacionadas al procesamiento de audio aun son un campo de estudio reciente para estas tecnologías, en donde todavía se presentan obstáculos para lograr una implementación plena de estas técnicas como por ejemplo: la falta de bases de datos masivas de señales acústicas, la selección de una manera de representación óptima de las señales que permita explotar sus características intrínsecas, las maneras de medir el rendimiento de los procesos, entre otros.

Por este motivo, esta investigación pretende realizar un análisis de esta problemática desde el punto de vista de la ingeniería de sonido, para comprender las limitaciones de los modelos actualmente utilizados en este campo de estudio, y poder aportar al progreso y mejora del rendimiento de dichos modelos. 

\subsection[Objetivos]{OBJETIVOS}
\subsubsection{Objetivo general}

El objetivo general de esta investigación es implementar un algoritmo de dereverberación de señales de voz a partir del uso de redes neuronales y algoritmos de aprendizaje profundo. 

\subsubsection{Objetivos específicos}

\begin{itemize}
    \item Realizar una revisión de las técnicas utilizadas para resolver el problema de dereverberación.
    \item Diseñar e implementar una estructura de red neuronal para dereverberación de señales de voz en lenguaje Python.
    \item Analizar las técnicas de pre y post procesamiento de datos, estudiando el impacto que tienen en el rendimiento del algoritmo.
    \item Optimizar el sistema propuesto, y comparar los resultados obtenidos con los modelos actuales de manera objetiva.
    \item Diseñar e implementar una interfaz gráfica en donde se permita visualizar los efectos del proceso de dereverberación aplicados a una señal particular. 
\end{itemize}

\subsection[Estructura de la Investigación]{ESTRUCTURA DE LA INVESTIGACIÓN}
En el capítulo 2 se presenta el estado del arte referido a las técnicas de dereverberación de señales del habla. 
En el capítulo 3 se detalla el marco teórico necesario para el seguiiento y comprensión de este trabajo. En este se abordan tres temáticas principales: La representación de señales de audio en el dominio espectral mediante la transformada de tiempo corto de Fourier, el concepto de reverberación y su relación con la respuesta al impulso, y por último la aplicación de redes neuronales convolucionales y algoritmos de aprendizaje junto con las principales técnicas de procesamiento.  
En el capítulo 4 se especifica de manera detallada la metodología seguida a lo largo de este trabajo, brindandose toda la información necesaria para replicar los experimentos realizados. 
En el capítulo 5 se presentan los resultados de los experimentos y se hace un análisis critico de los mismos. 
En el capítulo 6 se exponen las conclusiones generales del trabajo, y por último en el capítulo 7 se proponen lineas futuras de investigación relacionadas con el presente trabajo. 
\section[Estado del Arte]{CAPÍTULO 2:$\ \ \ \ $ESTADO DEL ARTE} 

En los últimos años se ha registrado un marcado desarrollo y progreso en el campo de el procesamiento de señales del habla. En este campo, la dereverberación ocupa un rol crucial, debido al impacto negativo que genera la presencia de reverberación en muchas aplicaciones del procesamiento de señales de habla. 

Los primeros enfoques que apuntaron a resolver el problema de la dereverberación se orientaron al modelado o registro de las respuestas al impulso y la estimación de filtros inversos a partir de estas \cite{filtros_inv}. Como el efecto de la reverberación en una señal se puede pensar como el resultado de una convolución entre una señal anecoica y una respuesta al impulso, este enfoque apunta a estimar la respuesta al impulso con el fin de poder generar un filtro inverso que permita realizar una deconvolución de la señal para poder revertir el efecto de la respuesta del recinto, recuperando la señal en su estado anecoico. Sin embargo, esta metodología presenta varios inconvenientes, como el hecho de considerar que las respuestas al impulso son lineales e invariantes en el tiempo, lo cual no siempre se cumple en la práctica \cite{LTI}, o bien el hecho de que la respuesta no siempre pueda ser deducida de manera directa y deba ser estimada. 


También surgieron trabajos enfocados en modelar matemáticamente la señal de habla anecoica  \cite{rabiner}. Algunos de estos trabajos consistían en estimar la señal de habla mediante predicción lineal, y calcular el residuo, el cual contiene información sobre la reverberación en la señal. Esta señal de residuo se utilizó para estimar filtros variantes en el tiempo que al aplicarse a la señal de habla lograban eliminar parte de la reverberación \cite{LPresiduo}. Otro enfoque consistió en utilizar múltiples transductores y aplicar técnicas de factorización matricial, como la descomposición en valores singulares (SVD), sobre las señales captadas \cite{multichannel}. Algunas características propias de las señales de habla, como la estructura armónica \cite{armonica} y el índice de modulación  \cite{mod}, fueron explotadas para eliminar los efectos de la reverberación. 

Posteriormente, se aplicó la idea de la sustracción espectral \cite{spect_subtrac} \cite{spect_subtrac2} que básicamente consiste en la estimación del espectro de potencia generado por la reverberación a partir de modelos estadísticos. En 2006, Wang et. al. aplicaron este enfoque combinado con el de la estimación de filtros inversos logrando presentar avances importantes en la efectividad de los algoritmos \cite{two_stage}.
 
 
El uso de máscaras binarias ideales en el dominio temporal-frecuencial para extraer las señales buscadas \cite{binarymask} es un enfoque muy utilizado, el cual tiene su origen en el campo del análisis computacional de escenas auditivas \cite{ASA}. Las máscaras se definen como ideales ya que su obtención requieren del conocimiento de la señal buscada y de la señal que interfiere. El uso de estas máscaras implica primero realizar una transformación de la señal de entrada de manera de trasladarla al dominio tiempo-frecuencia (por ejemplo un espectrograma, o un cocleograma) y luego asignarle a cada punto del espacio temporal-frecuencial un valor de 1 cuando su energía mayormente pertenece a la señal objetivo, y un valor de 0 en el caso contrario. Roman et. al. \cite{rev_mask} aplicaron este concepto para tratar el problema de dereverberación, donde se busca estimar la máscara binaria ideal tomando como señal objetivo la señal del habla en condiciones anecoicas y como interferencia a la parte reverberante. Para conseguir la dereverberación, este método requiere seleccionar de manera correcta parámetros como el punto desde el cual se distingue la parte temprana y tardía de la reverberación,y el nivel del umbral en base al cual se identifica a un punto específico como parte de la señal deseada o de la interferencia \cite{parametros}. Hazrati et al. \cite{hazrati} propusieron estimar la máscara binaria a partir de un parámetro dependiente de la varianza de la señal, la cual define un umbral adaptativo, obteniendo mejores resultados. 

A partir del año 2007, se comenzaron a aplicar redes neuronales en la tarea de dereverberación. Jin y Wang \cite{MLP} utilizaron perceptrones multicapa para estimar las mascaras binarias necesarias para la separación de la componente reverberante en una señal voz. La red neuronal aprende a estimar máscaras binarias a partir de la representación tiempo-frecuencia de la señal reverberada. Mas adelante, con el avance de los modelos de aprendizaje profundo, esta técnica se iría perfeccionando reflejándose en mejores resultados en la tarea de dereverberación. Los enfoques basados en la estimación de máscaras tuvieron variantes como por ejemplo la estimación de máscaras ideales reales \cite{cIRM}, mascaras ideales complejas \cite{IRM} y mascaras sensibles a la fase \cite{GAN}.
  
En 2014 Kun et al. \cite{ezeKun} proponen el uso de redes neuronales profundas para aprender el mapeo espectral de señales reverberantes hacia señales anecoicas. Esto quiere decir, en otras palabras, que se entrena una red neuronal profunda para que sea capaz de estimar el espectro anecoico a partir de la señal reverberada. Nuevamente se vuelve al planteo de la búsqueda del filtro inverso que permita la deconvolución de la señal reverberante para obtener su versión anecoica, pero en este caso se utilizaran redes neuronales para estimar ese filtro. 

Se han explorado una gran cantidad de arquitecturas y tipos de redes neuronales profundas. Las más utilizadas son las redes neuronales convolucionales (CNN), que surgieron del estudio de la corteza visual del cerebro y han sido muy exitosas en algunas tareas visuales complejas \cite{lagartija}.  Las CNN en general trabajan sobre espectrogramas de magnitud y tienen una estructura de codificador-decodificador \cite{FCN, rir_filtinverso}. Estas son eficientes en términos de parámetros, aunque requieren de una gran cantidad de capas (o profundidad). Esto se debe a que cada capa convolucional modela su entrada de forma local, con un campo receptivo limitado, y es necesario colocar muchas capas en serie para ampliar ese campo receptivo y abarcar la totalidad del espectrograma de entrada. 

Otro enfoque es utilizar redes neuronales que modelen el espectrograma de forma global, como es el caso de las redes recurrentes \cite{RNN}. Estas permiten aprovechar el contexto temporal de una secuencia de datos, por lo cual pueden extraer estructuras de corto y largo término, solucionando problemas asociados a las arquitecturas convolucionales como la discontinuidad entre espectrogramas.


También se implementaron sistemas que combinan dos o más arquitecturas. Por ejemplo, la combinación de redes convolucionales y redes recurrentes \cite{RNN+CNN}. De esta manera, la red convolucional permite analizar características locales en un contexto temporal fijo, y la red recurrente permite conservar información estructural de largo plazo. Otro ejemplo es la combinación de redes convolucionales y recurrentes con redes generativas adversarias (GAN) \cite{GAN}, lo cual produce una mejora en la calidad percibida del audio dereverberado generado. 

Gran parte de los trabajos procesan la magnitud del espectrograma, ignorando la información de fase. En estos casos, al realizar la inversión del espectrograma estimado, algunos sistemas \cite{ezeKun} utilizan el algoritmo de Griffin Lim \cite{griffinlim}, el cual permite invertir espectrogramas utilizando solamente su magnitud. Otros trabajos utilizan la fase original de la señal reverberada \cite{CNN, FCN, skip, rir_filtinverso}, lo cual es una solución sencilla aunque subóptima. Por último, en trabajos recientes la información de fase es utilizada por las redes neuronales, ya sea porque trabajan con el espectrograma complejo \cite{cIRM}, o porque modelan directamente la forma de onda \cite{hifiGAN}.
\section{Marco Teórico}


\subsection{Representación temporal y frecuencial de señales}

\subsection{Respuesta al impulso y reverberación}

Si en un recinto tengo una fuente y un micrófono captando a una cierta distancia de la fuente, las ondas sonoras que emite la fuente se reflejaran en las paredes del recinto y alcanzaran el micrófono inmediatamente después que la onda sonora directa. IMAGEN. Las reflexiones continúan ocurriendo, y cada instancia de reflexión supone una disminución de la energía sonora de la onda, principalmente causada por el efecto de absorción acústica de las superficies que producen las reflexiones.En un determinado tiempo, la energía sonora decaerá en todo el recinto hasta ubicarse por debajo del ruido de fondo. A este proceso se lo denomina reverberación. Al camino mas corto entre la fuente y el punto de captura se denomina camino directo, y a la relación de nivel entre la presión sonora que genera la onda propia del camino directo y la presión que genera el efecto de reverberación se lo conoce como relación directo-reverberado. 

Si el micrófono se ubica cerca de la fuente va a captar en mayor medida la señal correspondiente al camino directo, y una pequeña porción del sonido reverberado. Es decir, una relación directo-reverberado alta. A medida que el punto de captura se aleja de la fuente va a captar una menor cantidad del sonido correspondientemente al camino directo, mientras que el campo reverberado se mantendrá aproximadamente invariante. Esto se traduce en una disminución de la relación directo-reverberado. 

De esta manera, habrá una distancia específica para la cual el nivel de presión sonora generado por la fuente sera igual al nivel de presión sonora generado por el efecto de la reverberación. Esta distancia se conoce como distancia crítica. Esta depende tanto de las condiciones del recinto como de las características del micrófono. 

La función de transferencia entre la fuente emisora y el micrófono se define como la respuesta al impulso del recinto (RIR por sus siglas en inglés) y usualmente se denota como $h(t)$. Este sera diferente para cualquier punto en el espacio dentro del recinto. Haciendo un análisis temporal de una respuesta al impulso, podemos identificar 3 partes: En primer lugar el nivel de sonido directo (producido por la onda que viaja a través de camino directo), las reflexiones tempranas (cuyo limite temporal vendrá definido por las características propias de cada recinto) y por ultimo la cola reverberante. Se puede distinguir la parte de reflexiones tempranas y la cola reverberante partiendo de la suposición de que las reflexiones tempranas ocurren en un proceso determinístico, siendo altamente sensibles a pequeños cambios en la geometría del recinto, mientras que la cola reverberante es mas bien un proceso estocástico, y al depender de un mayor número de reflexiones no varia drásticamente frente a pequeños cambios de geometría.  

Idealmente, el micrófono captura una señal que corresponde a la convolución entre la respuesta al impulso del recinto y la señal fuente.

\begin{equation}
\label{eqn:pythagorean}
	$x(t) = h(t) * s(t)$



\end{equation} 


\subsection{Inteligibilidad y parámetros de calidad de percepción}

\subsection{Redes neuronales y algoritmos de aprendizaje}



\input{sections/4Disegno.tex}
%\input{sections/5Validacion.tex}
%\section{Análisis de los resultados}
%\section[Conclusiones]{CAPÍTULO 6:$\ \ \ \ $CONCLUSIONES} 

En este trabajo se implementó un algoritmo de aprendizaje profundo para la tarea de dereverberación de señales de habla. Se utilizó una estructura de tipo autoencoder con conexiones de salto, la cual se entrenó para estimar máscaras de amplitud que al aplicarse sobre un espectro de magnitud reverberado generen un espectro de magnitud dereverberado. Luego, para obtener la señal temporal a partir de la magnitud del espectrograma dereverberado, se utilizó la fase del espectrograma reverberado y se invirtió mediante la transformada inversa de Fourier y la técnica de suma y solapamiento.
 
Principalmente, se puso foco en analizar la influencia de ampliar el conjunto de datos de entrenamiento con técnicas de aumentación y síntesis de respuestas al impulso. También se analizó el efecto del ordenamiento de los datos de entrenamiento, y de la técnica utilizada para la inversión del espectrograma dereverberado.

Se pudo comprobar que una mayor diversidad de respuestas al impulso para la generación de los datos de entrenamiento genera mejores resultados, y que tanto las respuestas al impulso generadas como las aumentadas son útiles como método de aumentación de datos de entrenamiento. Por otro lado, se pudo comprobar que aplicar aprendizaje por currículum, ordenando los datos con un TR creciente, permite obtener mejores resultados de dereverberación.
 
Estos resultados son un aporte importante para el estudio de las tareas de dereverberación de audio, para las cuales las bases de datos de respuestas al impulso existentes son escasas, poco diversas y en ocasiones difíciles de conseguir o muy costosas.
%\input{sections/Lineas_futuras.tex}

\newpage

\printbibliography[heading=bibintoc,title={Bibliografia}]

%\appendix
%\section{}
%\section{Proceso de aumentación de $T_{60}$}
asdasd




\end{document}
